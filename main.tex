\documentclass[fleqn,10pt]{wlscirep}

\usepackage{graphicx}

%\begin{figure}
%\centering
%\includegraphics[width=0.75\linewidth]{stream.jpg}
%\caption{Caption. \label{label}}
%\end{figure}

\title{Hub.core Contracting Services}

\begin{document}

\flushbottom
\maketitle

\thispagestyle{empty}

\section{Quick Summary}

\begin{itemize}
    \item The Hub.core is a subsection of the Bioinformatics Hub that provides ad hoc analysis services
    \item Feasible projects are performed on a first come first serve basis in order of a project queue
    \item Projects are scoped to at most 40 hours of work, charged on a fee-per-project basis
    \item Financial contributions from subsidiary BU groups will subsidize fee rates for member researchers
    \item Hub personnel involved in providing analysis shall not be listed as authors on any resulting publications
\end{itemize}

\section{Introduction}

Official Hub projects are the primary vehicle for collaborations and scientific output of the BU Bioinformatics Hub.
These projects closely integrate Hub resources with the collaborating lab's research approaches and goals, and have well defined but open ended scope requiring significant time, attention, and effort.
While these projects are highly effective at advancing research, the time scale and amount of effort for these projects is sometimes greater than is required for specific bioinformatic research needs.
The Hub is occasionally contacted by researchers who have generated a single dataset as part of a larger research project and need focused support in the analysis of the data, but do not anticipate needing additional, long term analytical support.
The official Hub project model does not address this need, so an additional set of resources, called the Hub.core, has been established.
The purpose of the Hub.core is to provide ad-hoc analytical services for datasets on a fee-per-project based model.
Project fees may be subsidized using contributions from other sources of sustained support specifically devoted to Hub.core projects, depending on the details of the project.

\section{Organization}

Hub.core services are provided as a subset of the overall capabilities of the Hub.
Each Hub Analyst position has a maximum percent effort devoted to Hub.core work (currently 20\% FTE).
A Hub Scientist oversees the contact, ingestion, management, and completion of each Hub.core project, determined on a case-by-case basis according to bandwidth and expertise.

\section{Project Policies}

\subsection{Priority}

All new projects are entered into a project queue and assigned one of two priorities.
Initial, self contained projects (i.e. projects that were not split into two or more serial projects) are assigned priority 1 and are executed in the order in which they were accepted.
Each researcher may have only one priority 1 project in the queue at a time.
Subsequent projects submitted for each researcher are assigned priority 2.
A priority 2 project becomes a priority 1 project when either:

\begin{itemize}
    \item Three other priority 1 projects have superseded it due to priority
    \item It reaches the top of the project queue
\end{itemize}
This priority system is designed to ensure Hub.core services are shared equitably among researchers.

\subsection{Ingestion}

The project ingestion process begins when the Hub is contacted by a researcher with an immediate, concrete analytical need.
The managing Hub Scientist will discuss the goals and analytical needs of the project with the researcher to determine feasibility and level of effort required.
Feasibility is determined based on matching the tools and techniques required by the analysis and the computational capabilities of the Hub at the time of contact.
A project may be deemed unfeasible as a Hub.core project if it requires the analysis of datatypes with which the Hub does not have adequate experience.
In such cases, the Hub will attempt to find outside resources that may meet the analytical needs of the project.

If a project is assessed as feasible, the managing Hub Scientist will work to define specific deliverables and an estimated timeline for the project with the contacting researcher.
Deliverables are discrete, precisely defined analysis outputs, e.g. a spreadsheet with differentially expressed genes, histograms or scatter plots of underlying data, etc.
The deliverables and accompanying description are designed to provide the contacting researcher with enough information for them to interpret and translate the findings into actionable hypotheses or results with their own resources.
After the deliverables have been produced and delivered, additional interpretation services and support is not guaranteed, as Hub.core resources may need to be allocated to other projects in the queue.

The set of analyses and deliverables is used to determine an estimated timeline and number of hours of work required to complete the project.
Hub.core projects are of limited scope, and should generally require no more than 40 hours of hands-on time to complete (computation time, e.g. running analysis on a cluster overnight, does not count as hands-on time).
Projects requiring more hands-on time than 40 hours may be split into separate, serial projects that are entered into the project queue with priority 2 (see below).
The project description, methodology, deliverables, timeline, and expenses will be written into a structured statement of work document and shared with the researcher for feedback and approval.

\subsection{Execution}

In general, an analyst is limited to focusing on at most one active project at a time, namely the project currently on the top of the project queue.
When an analyst begins executing a project, the project is removed from the queue and other projects are shifted and reprioritized appropriately. 
The analyst then executes the project as described in the statement of work document until the deliverables have been completed.
During execution, the managing Hub Scientist will oversee progress and facilitate communication with the researcher should questions or needed clarifications arise.
When all of the deliverables are complete, the managing Hub Scientist will facilitate the communication and delivery of the results to the researcher and the analyst will move on to the next project in the project queue.

Substantive preparation of manuscripts as a result of Hub.core projects is not included as part of this service offering.
The managing Hub Scientist will provide textual description of the methods with appropriate references for inclusion in manuscripts, as well as any publication quality figures specified as deliverables.
The Hub follows the \href{http://www.icmje.org/recommendations/browse/roles-and-responsibilities/defining-the-role-of-authors-and-contributors.html}{authorship guidelines} described by the International Committee of Medical Journal Editors.
As Hub.core projects do not involve experimental design or the drafting or review of intellectual contributions in the final manuscript, no authorship rights are given to the Hub personnel who work to complete these projects.
An acknowledgement of the Hub and personnel who contributed to the work is sufficient.

\subsection{Fee Schedule}

Fees are calculated using the estimated project completion time at an hourly base rate of \$90 per project hour, or a maximum \$3600 per project.
This fee covers:

\begin{itemize}
    \item Hub Analyst salary + overhead (full time)
    \item Hub Scientist management effort + overhead (1/4 time)
    \item Infrastructure maintenance costs (SCC data storage, compute costs, AWS infrastructure costs, etc)
    \item Hub overhead costs (administration, workstation equipment, etc)
\end{itemize}

If subsidiary organizations within BU (e.g. departments) provide sustained support to the Hub, the rate above may be subsidized commensurately for members of those organizations.

\end{document}